%---------------------------------------------------------------------------%
%->> Backmatter
%---------------------------------------------------------------------------%
\chapter[致谢]{致\quad 谢}\chaptermark{致\quad 谢}% syntax: \chapter[目录]{标题}\chaptermark{页眉}
%\thispagestyle{noheaderstyle}% 如果需要移除当前页的页眉
%\pagestyle{noheaderstyle}% 如果需要移除整章的页眉

回首求学历程,感慨颇多。在本文完成之际,作者谨向给予我指导、关心、支持、帮助和点拨我的老师、同学和亲人们致以衷心的感谢!

首先, 我要衷心地特别感谢我的导师陈洪波教授。 毫无疑问此论文它的前提始终无法离开我的导师陈洪波教授的支持。 我要特别感谢陈教授为我在攻博期间的科研工作提供了极具优越的平台和条件; 我要特别感谢陈老师在攻博开题阶段以及在北京联培期间为我的研究给予的关切和指导; 更重要的是, 特别感恩陈老师对于团队以及我个人的教诲都始终站在国家、时代、人民的高度。 这样的言说还不仅仅只是停留在感激和感恩的层面, 更重要的恰恰是由于陈老师的点拨和教诲让我真切的意识到我们必须要在思想的事业上有最起码的路标, 而此路标的建构就无论如何都离不开这四年期间陈老师给我的指导和点拨。 

同时, 这篇论文得以可能的前提也与我在北京联合培养期间的导师陈小前研究员和姚雯研究员的支持密不可分。 感谢姚老师。 姚老师提供了极具前沿的课题, 本论文应用研究部分大都是在姚老师的课题下展开的, 并且也是在此课题的探索中启发了我必须批判地直面技术本身。 诚恳说来, 这样的课题要求已经在逻辑上规定了我本人进一步追问学习问题的一些方向。

感谢庄学彬副教授。 我攻博期间学术的开端是庄学彬老师提供的, 庄老师提供的前沿科研项目为我的研究提供了重要的应用背景并且开阔了学术眼界。 在庄老师的指导下我也能有机会深入理解一些课题方向并且有机会进入科研项目的语言体系中去。

感谢国防科技大学郭得科教授对我本人的关心和鼓励, 感谢郭老师在军科期间专门问询我本人的情况。感谢张英朝教授在我科研初期给予的点拨和鼓励。感谢国防科技大学任棒棒博士给我提供的帮助。

感谢兰州大学李周平老师提示我进入一致性预测研究领域,感谢李老师在我攻博期间对我特别的关怀和问询。感谢兰州大学李宪越老师给予我的提携和帮助, 有幸和李老师相识是我的福气。

我要特别感谢Vladimir Vapnik教授和Vladimir Vovk教授。 感谢Vladimir Vapnik教授在统计学习理论上给我额外重要的点拨。 感谢 Vladimir Vovk教授对我的鼓励。 正如Vladimir Vapnik教授所言, VC理论是思想的事业, 而思是我们这个时代的当务之急。我是在颇为不安地反思如何真切地辨明统计学与机器学习——最初是基于神经网络相关的机器学习——这两种知识类型的本质区别而展开研究。在学习理论中如何找出那条“林中路”是我一直以来在追问但又迟迟没解决的问题。 我一直是在统计学的框架下展开可信学习问题, 在这个框架下我研究的内容最后归结于如何获得有效的p-值。 直到2020年10月开始关注一致性预测方法, 因为这个方法能够自动保证p-值有效。 可以想见, 在先前那两三年的疑虑和困惑的停留中, 一致性预测方法为我开启了一种格外明朗的光照, 此光照起初在技术上的炫目让我深为服膺。 也正是在深入到一致性预测方法细节研究的时候——即如何理解非一致性得分函数——我真正开始阅读VC理论。 只有从存在论基础上深入地理解Vapnik统计学习理论哲学,不满足于当前主流受过逻辑训练心智给出的方案,关于学习理论的真正内容才有可能牵引着同我们真正照面。本文的第二章尝试着精要地给出这方面的一些结论。
%在我看来, Vapnik统计学习理论哲学唯有通过阅读Vapnik原著才能得以可能, 但是真正读懂Vapnik原著又需要读者自己发动革命。 这就给我们一个很有意思的循环: 要读懂Vapnik统计学习理论就需要Vapnik统计学革命思想, 而Vapnik统计学革命思想又在Vapnik统计学习理论中。 正就是在这样一种理解下我开始阅读Vapnik的原著, 通过阅读原著我的思想发生了彻底的转变。 Vapnik发动的统计学革命虽然在计算机领域的前沿研究中都得到了技术上的认可——这种认可更多由计算机领域的专家们以含蓄的技术阐述言说——但是在哲学的基础方面却并没有得到澄清, 甚至并未被真正研究过。 这里所说的基础方面最终涉及Vapnik统计学习理论哲学的存在论(ontology)根基。 如果这一根基本身依然停留在晦暗之中, 那么对于Vapnik统计学习理论的任何一种技术上的阐述就会是薄弱的和内在动摇的。 为了从先前知性的理解困境中摆脱出来, 我开始重新深入到VC理论的存在论基础上展开, 并对这一基础做出具有原则高度的阐释。 
%它的形式似乎基于统计学辩证法的, 的确我们需要给出统计学辩证法; 但真切的说来它的内容是存在论的, 是以重释VC理论存在论为枢轴的。 所以这就使得在内容上首先一定是批判的: 即不满足于当前主流受过逻辑训练心智给出的方案。 同时, 又是探索性的, 即试图在存在论的视域中来阐说VC理论。 

感谢在广州、北京学习期间遇到课题组的老师和同学的帮助, 感谢马才碑、马欣煜、王东华、王帅、王红亮、王超然、王晶、牛犇、左源老师、石涛、付鹏、包凯瑞、冯妤婕、朱效洲老师、刘书磊、刘旭、孙家亮、李桥、李超、肖叶、何雨昕、张云阳、张若凡、张泽雨、张晨、陈献琪博士、武领华、苗青、林子健、林煜龙、尚天赐、罗加享、郑小虎博士、项子雪、赵啸宇、姚炜杰、夏宇峰、唐贵建、彭兴文、曾小慧、曾昆、谢礼伟、谢扬帆、蔡鑫。

感谢南方科技大学的何帅达博士和兰州大学的刘倩倩博士提供的帮助和在前期展开关于统计学交流时提供的便捷。 感谢Wang Jie同学在关于存在论问题阐述时给出的积极反馈。

深深地感谢我的父亲和母亲。在我成长的每一个关键时期,都倾注了他们力所能及的最大支持。养儿方知父母恩。这样切近的体会,对语言所指和能指的真切把握是我自己成为父亲后方才领悟到的。

最后, 特别感谢爱人徐庆龄给予家庭的无尽付出。我唯有真切的进入真正的思以及勤勉的工作才是对家人默默支持最大的回报。 可以断言, 这也是家人所乐见的真正的自己。

\hfill \textit{席泽璞}

\hfill 2023年5月于广州中山大学


\chapter{作者攻读学位期间发表的学术论文与研究成果}
%
%\section*{作者简历:}
%
%席泽璞, 甘肃会宁人。 中山大学系统科学与工程博士研究生。 
%
%\begin{enumerate}
%\item[•] 2011年9月 - 2015年7月, 成都理工大学
%\item[•] 2015年9月 - 2018年7月, 兰州大学
%\item[•] 2019年9月 - 2023年7月, 中山大学
%\item[•] 2020年12月 - 2023年7月, 军事科学院国防科技创新研究院\quad 联合培养 
%\end{enumerate}

\section*{已发表文章}
\begin{enumerate}
\item 
\textbf{Zepu Xi}, Hongbo Chen, Xiaoqian Chen, Wen Yao. 
Conformal Prediction Enhanced SVMs for Swarm Behavior Classification. 
International Joint Conference on Neural Networks (IJCNN). 2023. (人工智能学术A类会议, CCF-C, 第一作者) 
\item 
\textbf{Zepu Xi}, Hongbo Chen, Xiaoqian Chen, Wen Yao. 
Swarm Behavior Recognition: Martingales Protected Paradigm. 
International Joint Conference on Neural Networks (IJCNN). 2023. (人工智能学术A类会议, CCF-C, 第一作者)
\item 
\textbf{Zepu Xi}, Hongbo Chen, Xiaoqian Chen, Wen Yao. 
Distributions-free Martingales Test Distributions-shift. 
International Neural Network Society Deep Learning Innovations and Applications Workshop (INNS DLIA 2023), 2023. (第一作者)
\item 
\textbf{Zepu Xi}, Xuebin Zhuang, Hongbo Chen. 
Conformal Prediction for Hypersonic Flight Vehicle Classification. 
Proceedings of Machine Learning Research (PMLR). 2022. (第一作者)
\item 
\textbf{Zepu Xi}, Hongbo Chen, Xiaoqian Chen, Wen Yao. 
Mondrian Conformal Prediction of Boosting for Swarm Behavior Recognition. 
2022 IEEE International Conference on Unmanned Systems ({IEEE ICUS}). 2022. (EI, 第一作者)
\item 
\textbf{Zepu Xi}. 
Distributions-free Martingale Test Distributions-shift for Swarm Behavior Prediction. 
IEEE WCCI 2022, 2022 International Joint Conference on Neural Networks (WCCI-IJCNN). 2022. (人工智能学术A类会议, CCF-C, 第一作者)
\item 
\textbf{Zepu Xi}, Xuebin Zhuang, Hongbo Chen. 
Post-selection Inference, with Application to Autoregressive Processes. 2021. 
International Conference on Mechanical, Aerospace and Automotive Engineering (CMAAE), 2021. (EI, 第一作者)
\item 
Kun Zeng, Xuebin Zhuang, Yangfan Xie, \textbf{Zepu Xi}. 
Hypersonic Vehicle Trajectory Classification Using Improved CNN-LSTM Model. 
2021 IEEE International Conference on Unmanned Systems (IEEE ICUS). 2021. (EI, 第四作者)
\item 
Yangfan Xie, Xuebin Zhuang, \textbf{Zepu Xi}, Hongbo Chen. 
Dual-Channel and Bidirectional Neural Network for Hypersonic Glide Vehicle Trajectory Prediction. 
IEEE Access, 2021. (SCI, 第三作者)
\end{enumerate}

%\section*{在学期间参加国际学术会议}
%\begin{enumerate}
%\item Conformal Prediction for Hypersonic Flight Vehicle Classification. 一致性和概率预测与应用大会(COPA2022). 主席: Vladimir Vovk. 英国, 布莱顿
%\item Distributions-free Martingale Test Distributions-shift for Swarm Behavior Prediction. IEEE计算智能学会(WCCI2022), 国际神经网络联合大会(IJCNN2022). 意大利, 帕多瓦
%\item Post-selection Inference, with Application to Autoregressive Processes. CMAAE2021. 中国香港
%\end{enumerate}

\section*{已投递文章}
\begin{enumerate}
\item 
\textbf{Zepu Xi}, Hongbo Chen, Xiaoqian Chen, Wen Yao. 
Online Inferring Swarm Behavior Using Privileged Information Learning. 
Engineering Applications of Artificial Intelligence (SCI, 在审, 第一作者)
\item 
\textbf{Zepu Xi}, Hongbo Chen, Xiaoqian Chen, Wen Yao. 
Comformal Prediction for Reliable Swarm Behavior Classification. 
Engineering Applications of Artificial Intelligence (SCI, 在审, 第一作者).
%\item \textbf{Zepu Xi}. Predicates Learning. (Submitted)
\end{enumerate}

%\section*{专利}
%\begin{enumerate}
%\item 庄学彬, 谢扬帆, 陈洪波, \textsf{席泽璞}, 曾昆. 一种基于双通道双向神经网络的飞行器轨迹预测方法[P]. 发明专利: CN112859898A, 2021-05-28. (已授权)
%\end{enumerate}
%
%
%
\section*{译著}
\begin{enumerate}
\item Vladimir N. Vapnik. 基于经验数据的相依关系估计[M]. 席泽璞, [译]. 第三版 (860+页), 2023. (第二次校稿中)
\end{enumerate}

\section*{参加的科研项目情况:}
\begin{enumerate}
\item 基于因果推断的XXX技术研究. 国防科技创新特区项目, 2022-2023年, 核心参与
\item 基于XXX技术研究. 国防科技创新研究院项目, 2021-2022年, 参与
\item 基于XXX信息的XXX预测技术研究. 国防科技创新特区项目, 2020-2021年, 核心参与
\item XXX系统韧性建模与评估研究. 装备预先研究项目, 2020-2021年, 核心参与
\end{enumerate}

\cleardoublepage[plain]% 让文档总是结束于偶数页,可根据需要设定页眉页脚样式,如 [noheaderstyle]
%---------------------------------------------------------------------------%
