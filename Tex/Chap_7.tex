\chapter{总结与展望}
\label{chapter:conclusion}

\section{论文工作总结}

鉴于复杂无人机集群行为通常和无人机集群任务分配具有强关联性, 本文基于无人机集群位置信息, 借助机器学习算法反演无人机集群行为, 并结合实际工程中无人机集群行为分类问题特点, 开展了四个方面的研究, 主要工作总结如下:

\begin{enumerate}
\item 研究无人机集群行为分类不确定量化方法。

提出一种使用一致性预测框架为无人机集群行为分类提供不确定量化研究解决方案。一致性预测方法提供两种预测模式, 其一, 置信预测, 此模式为机器学习算法输出提供一种无分布假设的不确定量化方法。 其二, 集合预测, 此模式为复杂无人机行为提供多类别的集合映射模式。试验结果表明, 所提方法可以为机器学习算法提供有效统计推断, 为解决复杂无人机集群行为分类问题中不确定量化研究提供一种解决方案。
\item 研究无人机集群行为数据本身的分布漂移检测。

提出一种使用鞅序列方法来实现针对无人机集群行为数据本身的分布漂移检测问题。 所提方法可建立在机器学习算法之上, 仅需要比较鞅序列最终值是否大于给定阈值就可以判定数据是否发生分布漂移, 为高维复杂无人机集群数据的分布漂移检测难题提出一种求解方法。试验结果表明, 所提方法能够实现对无人机集群行为数据分布漂移检测提供端到端的高效解决方案。
\item 研究使用鞅保护分布方法来改进底层机器学习算法的预测性能。

提出一种使用鞅保护分布算法(此方法融合了分布漂移信息)来增强底层分类算法的预测能力。 将基于鞅序列分布漂移检测理论和机器学习算法相结合, 改进常规机器学习算法预测性能。试验结果表明, 提出的鞅保护分布算法对大多数机器学习模型都有明显提升效果。
\item 研究小样本下高精度无人机集群行为分类方法。

提出一种使用特权信息学习范式实现小样本下高精度预测的方法。将无人机集群的整体描述作为特权信息纳入机器学习模型, 使得在小样本下达到高精度预测的目标。试验结果表明, 使用特权信息学习模型能够明显提升学习算法收敛速度, 使无人机集群行为分类在小样本下也可以达到较高精度的预测效果。
\end{enumerate}

\section{论文创新点}

本论文主要创新点如下:

\begin{enumerate}
\item 针对无人机集群行为分类, 提出采用一致性预测框架为机器学习算法输出结果提供不确定量化。应用蒙德里安一致性预测实现无人机集群行为分类不确定量化结果, 给出置信预测和集合预测两种不确定量化解决方案。
\item 针对实际工程应用需考虑数据本身分布的要求, 提出无分布假设的分布漂移检测方案。设计考虑数据本身分布信息的鞅保护分布算法, 改进底层机器学习算法预测性能, 为无人机集群行为分类问题提供一种新的高精度预测解决方案。
\item 提出使用特权信息模型加速分类算法收敛速度。设计针对无人机集群行为的两种整体描述信息, 验证整体描述作为特权信息的有效性, 实现小样本高精度无人机集群行为分类目标。
\end{enumerate}

\section{论文工作展望}

在本文已有研究工作基础上, 下一步工作可在以下几个方面展开:

\begin{enumerate}
\item 一致性预测框架为机器学习算法提供不确定量化的研究中, 可信度的效率与选取的非一致性得分函数有关。因此, 后续研究可以考虑如何提供更加高效的非一致性度量函数。
\item 无分布假设的鞅序列分布漂移检测研究中, 鞅理论提供的拒绝零假设证据建立在底层算法基础之上, 是一种依概率收敛的应诺式结果。因此, 后续研究可以探究如何改进底层学习算法。
\item 使用特权信息学习模型研究中, 模型预测效果与研究者提供的整体描述信息密切关联。因此, 后续研究可以探讨如何提供具有“先入之见”的特权信息, 以减少人工试错成本。
\end{enumerate}